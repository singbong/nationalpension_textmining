\documentclass[a4paper,landscape]{article}
\usepackage[margin=0.3cm]{geometry}
\usepackage{kotex}
\usepackage{tikz}
\usetikzlibrary{shapes.geometric, arrows.meta, positioning, calc, fit, backgrounds, shadows.blur}

% ── 메인 컬러 팔레트 ──
\definecolor{cBlue}{HTML}{3B82F6}
\definecolor{cGreen}{HTML}{22C55E}
\definecolor{cRose}{HTML}{F43F5E}
\definecolor{cSlate}{HTML}{64748B}

% ── 수집 단계 ──
\definecolor{cB1}{HTML}{EFF6FF}
\definecolor{cB2}{HTML}{DBEAFE}
\definecolor{cB3}{HTML}{BFDBFE}
\definecolor{cB4}{HTML}{93C5FD}
\definecolor{cB5}{HTML}{60A5FA}

% ── 전처리 단계 ──
\definecolor{cG1}{HTML}{F0FDF4}
\definecolor{cG2}{HTML}{DCFCE7}
\definecolor{cG3}{HTML}{BBF7D0}
\definecolor{cG4}{HTML}{86EFAC}

% ── 분석 단계 ──
\definecolor{cR1}{HTML}{FFF1F2}
\definecolor{cR2}{HTML}{FFE4E6}
\definecolor{cR3}{HTML}{FECDD3}
\definecolor{cR4}{HTML}{FDA4AF}

% ── 노드 스타일 ──
\tikzset{
    box/.style={
        rectangle, rounded corners=6pt,
        minimum width=3.0cm, minimum height=1.1cm,
        text width=2.7cm, align=center,
        font=\footnotesize, inner sep=4pt,
        blur shadow={shadow blur steps=6, shadow xshift=0.4pt, shadow yshift=-0.8pt, shadow blur radius=2.5pt, shadow opacity=12}
    },
    sbox/.style={
        rectangle, rounded corners=5pt,
        minimum width=2.6cm, minimum height=0.9cm,
        text width=2.3cm, align=center,
        font=\scriptsize, inner sep=3pt,
        blur shadow={shadow blur steps=5, shadow xshift=0.3pt, shadow yshift=-0.6pt, shadow blur radius=2pt, shadow opacity=10}
    },
    accent/.style={
        rectangle, rounded corners=6pt,
        minimum width=3.2cm, minimum height=1.2cm,
        text width=2.9cm, align=center,
        font=\footnotesize\bfseries, inner sep=5pt,
        blur shadow={shadow blur steps=8, shadow xshift=0.5pt, shadow yshift=-1pt, shadow blur radius=3.5pt, shadow opacity=18}
    },
    tribox/.style={
        rectangle, rounded corners=5pt,
        minimum width=2.2cm, minimum height=1.0cm,
        text width=1.9cm, align=center,
        font=\scriptsize, inner sep=3pt,
        blur shadow={shadow blur steps=5, shadow xshift=0.3pt, shadow yshift=-0.6pt, shadow blur radius=2pt, shadow opacity=10}
    },
    arr/.style={-Latex, line width=0.6pt, color=cSlate!60},
    darr/.style={-Latex, dashed, line width=1.1pt, color=cSlate!45},
    parbox/.style={
        rectangle, rounded corners=5pt,
        minimum width=7cm, minimum height=0.9cm,
        text width=6.5cm, align=center,
        draw=cBlue!35, dashed, line width=0.5pt,
        fill=cB1, font=\scriptsize, inner sep=4pt
    }
}

\begin{document}
\pagestyle{empty}

\centering
\begin{tikzpicture}

    % ============================================================
    %  1단계 : 데이터 수집  (중심 X = -10)
    % ============================================================

    \node[parbox] (par) at (-10, 0) {멀티스레딩 (Threading) · 병렬 크롤링};

    \node[tribox, fill=cB2, draw=cBlue!20] (news) at (-13.7, -2) {네이버 뉴스\\Selenium};
    \node[tribox, fill=cB2, draw=cBlue!20] (blog) at (-10,   -2) {네이버 블로그\\Selenium};
    \node[tribox, fill=cB2, draw=cBlue!20] (cafe) at (-6.3,  -2) {네이버 카페\\Selenium};

    \node[sbox, fill=cB3, draw=cBlue!25] (scroll) at (-10, -4) {무한 스크롤 처리\\(Infinite Scroll)};

    \node[tribox, fill=cB3, draw=cBlue!30] (exN) at (-13.7, -6.2) {newspaper3k\\본문 추출};
    \node[tribox, fill=cB3, draw=cBlue!30] (exB) at (-10,   -6.2) {BeautifulSoup\\본문 추출};
    \node[tribox, fill=cB3, draw=cBlue!30] (exC) at (-6.3,  -6.2) {BeautifulSoup\\본문 추출};

    \node[sbox, fill=cB4, draw=cBlue!35] (retry) at (-10, -8.4) {에러 URL 재시도\\(최대 3회)};

    \node[accent, fill=cB5, draw=cBlue!45, text=white] (pkl) at (-10, -10.4) {Pickle 저장\\(raw\_data/*.pkl)};

    % 화살표
    \foreach \n in {news,blog,cafe} { \draw[arr] (par) -- (\n); }
    \foreach \n in {news,blog,cafe} { \draw[arr] (\n) -- (scroll); }
    \draw[arr] (scroll) -- (exN);
    \draw[arr] (scroll) -- (exB);
    \draw[arr] (scroll) -- (exC);
    \draw[arr] (exN) -- (retry);
    \draw[arr] (exB) -- (retry);
    \draw[arr] (exC) -- (retry);
    \draw[arr] (retry) -- (pkl);

    % 경계 박스
    \begin{pgfonlayer}{background}
        \node[rectangle, rounded corners=12pt, draw=cBlue!30, line width=1pt,
              fill=cBlue!4, inner sep=0.55cm,
              fit=(par)(news)(cafe)(exN)(exC)(retry)(pkl)] (P1) {};
    \end{pgfonlayer}
    \node[font=\large\bfseries, text=cBlue, above=0.15cm of P1.north] {데이터 수집};

    % ============================================================
    %  2단계 : 데이터 전처리  (중심 X = 0.5)
    % ============================================================

    \node[sbox, fill=cG1, draw=cGreen!20] (clean) at (0.5, 0)
        {한글 정제 · 정규표현식};

    \node[sbox, fill=cG2, draw=cGreen!25] (rmNa) at (0.5, -2)
        {결측치 제거\\(2글자 미만 / 공백)};

    \node[accent, fill=cG3, draw=cGreen!35] (komo) at (0.5, -4)
        {Komoran\\형태소 분석};

    \node[sbox, fill=cG2, draw=cGreen!25] (noun) at (0.5, -6.2)
        {명사 추출\\(2글자 이상 필터링)};

    \node[box, fill=cG4, draw=cGreen!40] (stop) at (0.5, -8.4)
        {불용어 제거\\(사용자 정의+외부 파일)};

    % 화살표
    \draw[arr] (clean) -- (rmNa);
    \draw[arr] (rmNa) -- (komo);
    \draw[arr] (komo) -- (noun);
    \draw[arr] (noun) -- (stop);

    % 경계 박스
    \begin{pgfonlayer}{background}
        \node[rectangle, rounded corners=12pt, draw=cGreen!30, line width=1pt,
              fill=cGreen!3, inner sep=0.55cm,
              fit=(clean)(rmNa)(komo)(noun)(stop)] (P2) {};
    \end{pgfonlayer}
    \node[font=\large\bfseries, text=cGreen!80!black, above=0.15cm of P2.north] {데이터 전처리};

    % ============================================================
    %  3단계 : 데이터 분석  (중심 X = 10)
    % ============================================================

    \node[sbox, fill=cR1, draw=cRose!20] (freq) at (10, 0)
        {키워드 빈도 (Top 500)\\Counter.most\_common};

    \node[sbox, fill=cR2, draw=cRose!25] (sent) at (10, -2)
        {감성 분석\\(KNU 감성사전)};

    \node[sbox, fill=cR2, draw=cRose!25] (lda) at (10, -4)
        {토픽 모델링\\(Gensim LDA · 5개)};

    \node[box, fill=cR3, draw=cRose!35] (res) at (10, -6.2)
        {결과 저장\\(Pickle .pkl)};

    \node[accent, fill=cR4, draw=cRose!45, text=white] (viz) at (10, -8.4)
        {시계열 시각화\\(matplotlib)};

    % 화살표
    \draw[arr] (freq) -- (sent);
    \draw[arr] (sent) -- (lda);
    \draw[arr] (lda)  -- (res);
    \draw[arr] (res)  -- (viz);

    % 경계 박스
    \begin{pgfonlayer}{background}
        \node[rectangle, rounded corners=12pt, draw=cRose!30, line width=1pt,
              fill=cRose!3, inner sep=0.55cm,
              fit=(freq)(sent)(lda)(res)(viz)] (P3) {};
    \end{pgfonlayer}
    \node[font=\large\bfseries, text=cRose, above=0.15cm of P3.north] {데이터 분석};

    % ============================================================
    %  단계 간 연결
    % ============================================================
    \draw[darr]
        (pkl.east) -- node[above, font=\scriptsize\itshape, fill=white, inner sep=2pt, text=cSlate] {raw\_data} (clean.west);

    \draw[darr]
        (stop.east) -- node[above, font=\scriptsize\itshape, fill=white, inner sep=2pt, text=cSlate] {tokenized} (freq.west);

\end{tikzpicture}

\end{document}
